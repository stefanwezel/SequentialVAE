\documentclass{article} % For LaTeX2e
\usepackage{neurips,times}
\usepackage{hyperref}
\usepackage{url}
\usepackage{graphicx}
\usepackage{lipsum}

\usepackage[numbers]{natbib}
\setlength{\bibsep}{0.0pt}

\title{Unsupervised Learning of Disentangled and Interpretable Representations from Sequential Data\\ \vspace{0.5cm}\large{Report}}
\author{Stefan Wezel \\ stefan.wezel@student.uni-tuebingen.de \\4080589  \\ ML4S}

\newcommand{\fix}{\marginpar{FIX}}
\newcommand{\new}{\marginpar{NEW}}

\nipsfinalcopy

\begin{document}


\maketitle

\begin{abstract}
%Sequential data often has the intrinsic quality of containing information playing out on multiple time scales. Features can appear low frequencies and on high frequencies. 

%
%While Variational Autoencoders (VAE) have proven to be a successful methodology on i.e. image data, 



Information in sequential data is often distributed over multiple time scales.
While if viewed as a single signal, such data might appear noisy. However, patterns can emerge if temporal scales are viewed separately from one another.
\citet{hsu2017unsupervised} leverage this intrinsic structure to learn disentangled representations from sequential data in an unsupervised manner. They propose to factorize sequence level and segment level attributes into distinct latent subspaces. Architectural and sequence dependent priors create an inductive bias to encourage the proposed factorization. Here, we put their work into a formal context, explore the proposed methodology, and reflect critically on their work.
\end{abstract}

\section*{Introduction}
%Learning disentangled representations has long been a difficult problem in machine learning. 
Intuitively, disentangled representations are reflective of the underlying generating factors of observed data in thus they are encoded as separate latent subspaces. This notion is already present in classical factor analysis work, where it is referred to as independent component analysis (ICA) \cite{comon1992independent}.\\
However, many problems cannot be solved in linear fashion. The vast success of deep neural networks (DNN) can be largely attributed to the fact that they are very powerful non-linear function approximators. Thus, making them an promising method to solve the long standing problem of non-linear ICA.\\
Different methods have been proposed to learn such disentangled representations \cite{higgins2016beta, chen2016infogan, kulkarni2015deep} with varying success. Many of these works focus on image data. However, it has been shown by \citet{locatello2019challenging} that disentangled representations cannot by learned without introducing any kind of supervision or inductive biases. Sequential data, while having been explored less, despite offers inherent structure that can be exploited to construct inductive biases as has been proposed by \citet{hsu2017unsupervised}.\\
Besides technical challenges, this strain of research suffers from the lack of formally defined and agreed upon foundations. The very term of disentangled representations for example is often understood differently in between works. In the following section, we will use the definition, proposed by \citet{higgins2018towards} to put the work by \citet{hsu2017unsupervised} into formal context. 

%It would provide a remedy to problems such as uninterpretable representation.


%Since \citet{locatello2019challenging} proved that learning such representations without any supervision or inductive biases is impossible, many works have focused on 

%\lipsum[2-3] (Figure \ref{fig:figure_01}).

%\begin{figure}[h]
%    \includegraphics[width=\textwidth]{figures/figure1.png}
%    \caption{kjsdflkajsdkfjsd}
%    \label{fig:figure_01}
%\end{figure}

\section*{Theoretical Context}
\lipsum[4-6]\cite{Papamakarios2016, Ardizzone2019}

\subsection*{Group Theory}


\subsection*{Symmetries in Sequential Data}


\section*{FHVAE}
\lipsum[4-6]\cite{Cranmer2020}

\section*{Results}
\lipsum[7-12]\cite{Brehmer2020}

\section*{Discussion}
Further exploit the available data using cross reconstruction.

\section*{Conclusion}
\lipsum[1-2]

%\section*{Acknowledgments}
%\lipsum[1]

\bibliographystyle{unsrtnat}
\bibliography{refs}

\end{document}
