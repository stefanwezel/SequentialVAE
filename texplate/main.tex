\documentclass{article} % For LaTeX2e
\usepackage{neurips,times}
\usepackage{hyperref}
\usepackage{url}
\usepackage{graphicx}
\usepackage{lipsum}

\usepackage[numbers]{natbib}
\setlength{\bibsep}{0.0pt}

\title{Unsupervised Learning of Disentangled and Interpretable Representations from Sequential Data\\ \vspace{0.5cm}\large{Report}}
\author{Stefan Wezel \\ stefan.wezel@student.uni-tuebingen.de \\4080589  \\ ML4S}

\newcommand{\fix}{\marginpar{FIX}}
\newcommand{\new}{\marginpar{NEW}}

\nipsfinalcopy

\begin{document}


\maketitle

\begin{abstract}
%Sequential data often has the intrinsic quality of containing information playing out on multiple time scales. Features can appear low frequencies and on high frequencies. 

%
%While Variational Autoencoders (VAE) have proven to be a successful methodology on i.e. image data, 



Information in sequential data is distributed over multiple time scales.
While if viewed as a single signal, such data might appear noisy, patterns can emerge if we look at just one specific temporal scale.
\citet{hsu2017unsupervised} leverage this intrinsic structure to learn disentangled representations from sequential data in an unsupervised manner. They propose to factorize sequence level and segment level attributes into distinct latent subspaces.
Here, we put their work into a formal context, explore the proposed methodology, and reflect critically on their work.
\end{abstract}

\section*{Introduction}
\lipsum[2-3] (Figure \ref{fig:figure_01}).

\begin{figure}
    \includegraphics[width=\textwidth]{figures/figure1.png}
    \caption{kjsdflkajsdkfjsd}
    \label{fig:figure_01}
\end{figure}

\section*{Theoretical Context}
\lipsum[4-6]\cite{Papamakarios2016, Ardizzone2019}

\subsection*{Group Theory}


\subsection*{Symmetries in Sequential Data}


\section*{FHVAE}
\lipsum[4-6]\cite{Cranmer2020}

\section*{Results}
\lipsum[7-12]\cite{Brehmer2020}

\section*{Discussion}
Further exploit the available data using cross reconstruction.

\section*{Conclusion}
\lipsum[1-2]

%\section*{Acknowledgments}
%\lipsum[1]

\bibliographystyle{unsrtnat}
\bibliography{refs}

\end{document}
