% !TeX program = xelatex
\documentclass[9pt]{beamer}
\usepackage{xcolor}
\definecolor{orange}{HTML}{F67941}
\definecolor{red}{HTML}{AC454A}
\definecolor{brown}{HTML}{EAD296}
\definecolor{darkgrey}{HTML}{313630}
\usefonttheme{professionalfonts} % using non standard fonts for beamer
\usefonttheme{serif} % default family is serif
\usepackage{fontspec}
\usepackage{setspace}
\usepackage{natbib}
%\usepackage[T1]{fontenc}

\bibliographystyle{abbrv}
%\setmainfont{Liberation Serif}
%\setmainfont{Liberation Serif}
\setmainfont{Comfortaa}
%\usepackage[T1]{fontenc}

\setbeamercolor{frametitle}{bg=orange,fg=white}
\setbeamercolor{author in head/foot}{bg=orange,fg=white}

%\setbeamerfont{page number}{size=\Huge}

%\setbeamertemplate{itemize items}[circle]
\useinnertheme{circles}
\setbeamercolor{palette primary}{bg=orange,fg=white}
%\setbeamercolor{palette secondary}{bg=red,fg=white}
\setbeamertemplate{itemize item}{\color{darkgrey}$\circ$}
\setbeamercolor{structure}{fg=red} % itemize, enumerate, etc

%\setbeamercolor{section in head/foot}{bg=red}
\setbeamercolor{title}{fg=orange} %, bg=brown
\setbeamercolor{author}{fg=darkgrey}
\setbeamercolor{institute}{fg=darkgrey}
\setbeamercolor{date}{fg=darkgrey}
%\setbeamercolor{normal text}{fg=darkgrey}
\makeatletter
\setbeamertemplate{headline}{%
	\usebeamercolor[bg]{frametitle}\rule{\textwidth}{1cm}
}
\setbeamerfont{title}{size=\LARGE}
\setbeamerfont{institute}{size=\normalsize}
\renewcommand*{\bibfont}{\scriptsize}


\setbeamertemplate{frametitle}{%
	\vskip-1cm%
	\begin{minipage}[c][\headheight][c]{\textwidth}%
		\usebeamerfont{frametitle}
		\strut\insertframetitle\par
		{%
			\ifx\insertframesubtitle\@empty%
			\else%
			{\usebeamerfont{framesubtitle}\usebeamercolor[fg]{framesubtitle}\strut\insertframesubtitle\par}%
			\fi
		}%      
		\vspace*{0.05cm}
	\end{minipage}%
	\vskip-0.1em
}
%\setbeamertemplate{footline}{%
%	\leavevmode%
%	\hbox{\begin{beamercolorbox}[wd=\paperwidth,ht=4.5ex,dp=3.125ex]{author in head/foot}%
%			\usebeamerfont{author in head/foot} bar
%	\end{beamercolorbox}}%
%	\vskip0pt%
%}
\makeatother


\title{Unsupervised Learning \\
	of Disentangled and Interpretable Representations from Sequential Data}
\author{Wei-Ning Hsu, Yu Zhang, and James Glass\\Talk by Stefan Wezel}
\institute{Explainable Machine Learning}

\date{\today}


%\setbeamertemplate{sidebar right}{}
%\setbeamertemplate{footline}{%
%	\hfill\usebeamertemplate***{navigation symbols}
%	\hspace{1cm}\insertframenumber{}}
\setbeamerfont{page number in head/foot}{size=\small}
    \setbeamertemplate{footline}{%
	\raisebox{5pt}{\makebox[\paperwidth]{\hfill\makebox[10pt]{\scriptsize\insertframenumber}}}}
\setbeamertemplate{navigation symbols}{}
%\onehalfspacing
\setstretch{1.3}
\begin{document}
	

\setbeamercolor{background canvas}{bg=white}
\setbeamercolor{normal text}{fg=darkgrey}
\usebeamercolor[fg]{normal text}
\begin{frame}[plain]
	\titlepage
\end{frame} 



\setbeamercolor{background canvas}{bg=white}
\setbeamercolor{normal text}{fg=darkgrey}
\usebeamercolor[fg]{normal text}
\setbeamertemplate{itemize item}{\color{darkgrey}$\circ$}
\begin{frame}
\frametitle{Overview}
%\framesubtitle{}
\begin{itemize}%\setlength\itemsep{1.5em}
	\item What is disentanglement (intuition)
	\item Why disentanglement
	\item Formal description of disentangled representations
	\item SequentialVAE
	\item Did they achieve disentanglement?
	\item Other approaches and challenges
\end{itemize}
\end{frame} 





\setbeamercolor{background canvas}{bg=white}
\setbeamercolor{normal text}{fg=darkgrey}
\usebeamercolor[fg]{normal text}
\setbeamertemplate{itemize item}{\color{darkgrey}$\circ$}
\begin{frame}
\frametitle{What is disentanglement?}
\framesubtitle{Intuition}
	\begin{itemize}%\setlength\itemsep{1.5em}
	\item ...
	\end{itemize}
\end{frame} 





\setbeamercolor{background canvas}{bg=white}
\setbeamercolor{normal text}{fg=darkgrey}
\usebeamercolor[fg]{normal text}
\setbeamertemplate{itemize item}{\color{darkgrey}$\circ$}
\begin{frame}
\frametitle{Why learn disentangled representations?}
\framesubtitle{Motivation}
\begin{itemize}%\setlength\itemsep{1.5em}
	\item ...
\end{itemize}
\end{frame} 





\setbeamercolor{background canvas}{bg=white}
\setbeamercolor{normal text}{fg=darkgrey}
\usebeamercolor[fg]{normal text}
\setbeamertemplate{itemize item}{\color{darkgrey}$\circ$}
\begin{frame}
\frametitle{What are disentangled representations formally? }
\framesubtitle{A field-trip to group theory}
\begin{itemize}%\setlength\itemsep{1.5em}
	\item ...
\end{itemize}
\end{frame} 

\setbeamercolor{background canvas}{bg=white}
\setbeamercolor{normal text}{fg=darkgrey}
\usebeamercolor[fg]{normal text}
\setbeamertemplate{itemize item}{\color{darkgrey}$\circ$}
\begin{frame}
\frametitle{Did they achieve disentanglement}
\framesubtitle{...}
\begin{itemize}%\setlength\itemsep{1.5em}
	\item With respect to a decomposition into two
\end{itemize}
\end{frame} 




\setbeamercolor{background canvas}{bg=white}
\setbeamercolor{normal text}{fg=darkgrey}
\usebeamercolor[fg]{normal text}
\setbeamertemplate{itemize item}{\color{darkgrey}$\circ$}
\begin{frame}
\frametitle{Challenges}
\framesubtitle{...}
\begin{itemize}%\setlength\itemsep{1.5em}
	\item If we really think about it, it is hard for us to define what a disentangled representation should actually be
	\item Precise biases of what the latent space should be decomposited into can be helpful as well as biases towards the 'form' of these latent subspaces
\end{itemize}
\end{frame} 

\end{document}
